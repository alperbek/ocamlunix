%------------------------------------------------------------------------------
% Copyright (c) !!COPYRIGHTYEAR!!, Xavier Leroy and Didier Remy.  
%
% All rights reserved. Distributed under a creative commons
% attribution-non-commercial-share alike 2.0 France license.
% http://creativecommons.org/licenses/by-nc-sa/2.0/fr/
%
% Translation by
%------------------------------------------------------------------------------

\chapter*{\ifhtml{\aname{htocmore}}Pour aller plus loin}
\addcontentsline{toc}{chapter}{\ifhtml{\ahrefloc{htocmore}}{Pour aller
plus loin}}
\label{sec/more}
\cutname{more.html}

Nous avons montr� comment utiliser les modules \libmodule{Sys},
\libmodule{Unix}, et \libmodule{Threads} d'{\ocaml} pour programmer
des applications qui interagissent avec le syst�me d'exploitation.

Ces biblioth�ques rel�vent les appels syst�me Unix les plus importants
au niveau du langage {\ocaml}. Au passage, certains de ces appels
syst�me ont �t� remplac�s par des fonctions de plus haut niveau, soit
pour faciliter la programmation, soit pour maintenir des invariants de
l'environnement d'ex�cution des programmes {\ocaml}.  En g�n�ral, cela
conduit � une �conomie dans l'�criture des applications.

Certaines fonctionalit�s du syst�me Unix ne sont pas accessibles au
travers des biblioth�ques pr�c�dentes, mais il est toujours possible
d'y acc�der directement via du code C.

Il existe aussi une biblioth�que Cash~\cite{Cash} d�di�e � l'�criture
de scripts en Ocaml. Cette biblioth�que compl�te le module \ml+Unix+
dans deux directions diff�rentes. D'une part, elle peut se voir comme
une couche au dessus du module \ml+Unix+ qui offre, en plus de
fonctions d�di�es � l'�criture de scripts, de nombreuses variations
autour des appels syst�mes d'\ml+Unix+, en particulier en ce qui
concerne la gestion des processus et des tuyaux.  D'autre part, elle
fournit quelques acc�s suppl�mentaires au syst�me Unix.
