%------------------------------------------------------------------------------
% Copyright (c) !!COPYRIGHTYEAR!!, Xavier Leroy and Didier Remy.  
%
% All rights reserved. Distributed under a creative commons
% attribution-non-commercial-share alike 2.0 France license.
% http://creativecommons.org/licenses/by-nc-sa/2.0/fr/
%
% Translation by Thaddeus Meyer
%------------------------------------------------------------------------------

\chapter{\label{sec/signals}Signals}
\cutname{signals.html}

Signals, or software interrupts, are external, asynchronous events
used to alter the course of a program.  These may occur at any
time during the execution of a program.  Because of this,
they differ from other methods of inter-process communication, where
two processes must be explicitly directed to wait for external messages;
for example, by calling \indexvalue{read} on a pipe (see 
chapter~\ref{sec/pipes}).

The amount of information transmitted via a signal is minimal, just
the type of signal, and although they were not originally
intended for communication between processes, they do make it
possible to transmit atomic information about the state of an
external entity (\eg{} the state of the system or another process). 

\section{Default behavior}

When a process receives a signal, there are four possible outcomes:
%
\begin{itemize}
\item 
The signal terminates the process.  Additionally, the system may write
an image of the process state in a core file (a \emph{core dump}, which
may be later examined with a debugger).
%
\item 
The signal suspends process execution, but retains it in memory.  The
parent process (usually the shell) is not terminated, and so may
choose to continue the process or restart it in the background by
sending the process additional signals.
%
\item Nothing occurs: the signal is completely ignored.
%
\item The signal triggers the execution of an associated function in the
receiving process.  Normal execution of the process resumes after the
function returns.
\end{itemize}
%

There are several types of signals, each associated with a particular event.  
Table~\ref{tab/signals} lists some of them with their default behaviors.
\begin{mytable}
\begin{tabular}{lll}
Name & Event & Default Behavior \\
\hline
\ml+sighup+ &
Hang-up (end of connection) &
Termination \\
\ml+sigint+ &
Interruption (\ml+ctrl-C+) &
Termination \\
\ml+sigquit+ &
Strong interruption (\ml+ctrl-\+) &
Term.\ \& core dump \\
\ml+sigfpe+ &
Arithmetic error (division by zero) &
Term.\ \& core dump \\
\ml+sigkill+ &
Very strong interruption (cannot be ignored) &
Termination \\
\ml+sigsegv+ &
Memory protection violation &
Term.\ \& core dump \\
\ml+sigpipe+ &
Writing to a pipe without readers &
Termination \\
\ml+sigalrm+ &
Timer interrupt &
Ignored \\
\ml+sigtstp+ &
Temporary halt (\ml+ctrl-Z+) &
Suspension \\
\ml+sigcont+ &
Resuming a stopped process &
Ignored \\
\ml+sigchld+ &
A child process died or was stopped &
Ignored \smallskip\\
\hline
\end{tabular}
\caption{Some signals and their default behaviors}
\label{tab/signals}
\end{mytable}

The signals received by a process come from several possible sources:
%
\begin{itemize}

\item The user may send signals via the keyboard.  By typing \verb'ctrl-C',
  the console operator sends the \ml+sigint+ signal to the processes
  controlled by her terminal (that were not already put in the background).
  In the same way, the \ml+sigquit+ signal is sent by typing \verb'ctrl-\'\footnote{These 
    are the default keystrokes, but it is possible to change them by 
    modifying the properties of the terminal, see section~\ref {sec/termio}.}.  
  When the terminal is closed (either through voluntary disconnection or owing to a disconnected network link), the \ml+sighup+ signal is sent.

\item The user may issue the shell command \ml+kill+.  This makes it possible 
  to send a specific signal to a specific process.  For example, 
  \ml+kill -KILL 194+ sends the \ml+sigkill+ signal to the process with id 194, which 
  causes the process to be killed.

\item Another program invoked the system call 
  \ml+kill+ (the preceding example being a specific case).

\item The system, for misbehaving processes.  For example, a process 
  attempting to divide by zero will receive a \ml+sigfpe+ signal.

\item The system, to notify a process that its execution environment has 
  been changed.  For example, when a child process terminates, its parent 
  will receive a \ml+sigchld+ signal.

\end{itemize}


\section{\label{sec/usingsignals}Using signals}

The system call \syscall{kill} makes it possible to send a
signal to a process.
%
\begin{listingcodefile}{tmpunix.mli}
val $\libvalue{Unix}{kill}$ : int -> int -> unit
\end{listingcodefile}
% 
The first parameter is the process id of the destination process and
the second the signal number to send. An error occurs if we attempt to
send a signal to a process not owned by the user.  A process may send
signals to itself.  When the \ml+kill+ system call returns, it is
guaranteed that the signal was delivered to the destination
process. If a process receives the same signal in rapid succession it
will execute the code associated with the signal only once.  Therefore
a program cannot count the number of times it receives a signal,
rather only the number of times it responds to it.

The system call \syscall{alarm} makes it possible to schedule
interruptions based on the system clock.
%
\begin{listingcodefile}{tmpunix.mli}
val $\libvalue{Unix}{alarm}$ : int -> int
\end{listingcodefile}
%
The call \ml+alarm s+ returns immediately but causes the \ml+sigalrm+ 
signal to be sent to the calling process at least \ml+s+ seconds later 
(note that there is no guarantee on the maximum wait time).  The call returns 
the number of seconds remaining to an alarm scheduled by a previous call.
If \ml+s+ is \ml+0+, the effect is simply to cancel an earlier alarm.

\section{Changing the effect of a signal}

The system call \syscall{signal} makes it possible to modify the behavior
of a process when it receives a signal of a certain type.
%
\begin{codefile}{tmpsys.mli}
type signal_behavior = Sys.signal_behavior
\end{codefile}
%
\begin{listingcodefile}{tmpsys.mli}
val $\libvalue{Sys}{signal}$ : int -> signal_behavior -> signal_behavior
\end{listingcodefile}
% 
The first argument is the signal number and the second argument a
value of type \libtype{Sys}{signal\_behavior}, indicates the desired
behavior for the signal. With:
\begin{mltypecases}
\begin{tabular}{@{}ll}
\ml+Signal_ignore+ & The signal is ignored. \\
%
\ml+Signal_default+ & The default behavior.\\
%
\ml+Signal_handle f+ & The function \ml+f+ is
invoked each time the signal is received.  
\end{tabular}
\end{mltypecases}

Forking a process with the system call \indexlibvalue{Unix}{fork}
preserves signal behavior: the initial definitions for the child are
those of the parent at the time when \ml+fork+ was executed.  The
\indexlibvalue{Unix}{execve} system call sets all the behaviors to
\ml+Signal_default+ except that signals ignored before are still
ignored afterward.

\begin{example}
Occasionally we wants to log-off or end a session while allowing 
background tasks (large calculations, \quotes{spyware} programs, \etc) 
to continue to run.  If this is desired, processes which normally
exit on receiving \ml+sighup+ (sent at the time the user disconnects)
should be prevented from doing so. The Unix command \ml+nohup+ does
exactly this:
\begin{lstlisting}
nohup cmd arg1 ... argn
\end{lstlisting}
executes the command \ml+cmd arg1 ... argn+ in a way unaffected by
the signal \ml+sighup+ (certain shells execute \ml+nohup+
automatically for all processes launched as background tasks).  Here's how
to implement this in three lines:
%
\begin{listingcodefile}{nohup.ml}
open Sys;;
signal sighup Signal_ignore;;
Unix.execvp argv.(1) (Array.sub argv 1 (Array.length argv - 1));;
\end{listingcodefile}
%
The system call \indexlibvalue{Unix}{execvp} preserves the fact that
\ml+sighup+ is ignored.
\end{example}

\begin{example} 
Carefully exiting when a program is misbehaving. For example,
a program like \ml+tar+ can try to save important information
in a file or destroy the corrupted file before terminating.  For this 
it is possible to include the following lines at the beginning of the program:
%
\begin{lstlisting}
signal sigquit (Signal_handle quit);
signal sigsegv (Signal_handle quit);
signal sigfpe  (Signal_handle quit);
\end{lstlisting}
%
where the function \ml+quit+ is of the form:
%
\begin{lstlisting}
let quit _ =
  (* Try to save important information in a file *);
  exit 100;;
\end{lstlisting}
\end{example}

\begin{example} 
Capturing user-initiated interruptions. Some interactive programs
need to return to a main control loop when a user
presses \ml+ctrl-C+.  For this we just need to raise an exception when the
\ml+sigint+ signal is received.
%
\begin{lstlisting}
exception Break;;
let break _ = raise Break;;
...
let main_loop () =
  signal sigint (Signal_handle break);
  while true do
    try (* Read and evaluate user commands  *)
    with Break -> (* Display "stopped" *)
  done;;
\end{lstlisting}
\end{example}

\begin{example} 
\label{ex/beep}
To carry out periodic tasks (animations, \etc) interleaved with
the execution of the main program.  For example, here is how
to create \quotes{beep} sounds every 30 seconds, regardless of
the activity of the main program (calculations or input/output).

\begin{lstlisting}
let beep _ = 
  output_char stdout '\007'; flush stdout; 
  ignore (alarm 30);;
...
signal sigalrm (Signal_handle beep); ignore (alarm 30);;
\end{lstlisting}
\end{example}

\subsection*{Checkpoints}

Signals are useful for asynchronous communication~---~indeed, it is
their raison d'\^etre~---~but this asynchronous nature also makes them
one of the major difficulties of system programming.

The signal handling function is executed asynchronously and thus
pseudo-concurrently with the main program of the process.  As signal
handling functions cannot return a value, they usually modify global
variables. This can result in race conditions between the signal
handler and the main program if they try to modify the same variable.
As explained in the next section one solution is to temporarily block
signals when this variable is accessed by the main program.

In fact, {\ocaml} does not treat signals in a strictly asynchronous
fashion.  On receiving a signal, {\ocaml} records the receipt of the
signal but the signal handling function will only be executed at
certain \emph{checkpoints}.  These are frequent enough to provide the
illusion of asynchronous treatment.  The checkpoints typically occur
during allocations, loop controls, or interactions with the system
(particularly system calls). {\ocaml} guarantees that a program that
does not loop, does not allocate, and does not interact with the
system will not interleave a the signal handler. As such writing an
unallocated value (integer, boolean, \etc{}~---~but not a float!) in a
reference cell cannot result in a race condition as described above.

\section{How to mask signals}

Signals may be blocked.  Blocked signals are not ignored, but put on 
standby, generally to be delivered later.  The 
\syscall{sigprocmask} system call makes it possible to change the mask
for incoming signals:
%
\begin{codefile}{tmpunix.mli}
type sigprocmask_command = Unix.sigprocmask_command
\end{codefile}
%
\begin{listingcodefile}{tmpunix.mli}
val $\libvalue{Unix}{sigprocmask}$ : sigprocmask_command -> int list -> int list
\end{listingcodefile}
% 
\ml+sigprocmask cmd sigs+ changes the list of blocked signals and
returns the list of signals that were blocked before the execution of
the function. This makes it possible to later reset the mask to its
previous state. The argument \ml+sigs+ is a list of signals and \ml+cmd+
a value of type \libtype{Unix}{sigprocmask\_command} which determines the
effect of the call:
\begin{mltypecases}
\begin{tabular}{@{}ll}
\ml+SIG_BLOCK+ & The signals \ml+sigs+ are added
to the list of blocked signals. \\
%
\ml+SIG_UNBLOCK+ & The signals \ml+sigs+ are removed
from signal mask. \\
%
\ml+SIG_SETMASK+ & The signals \ml+sigs+ are exactly the 
signals to be blocked.
\end{tabular}
\end{mltypecases}
%
A typical usage of \ml+sigprocmask+ is to mask certain
signals temporarily.
%
\begin{lstlisting}
let old_mask = sigprocmask cmd sigs in 
(* do something *)
let _ = sigprocmask SIG_SETMASK old_mark
\end{lstlisting}
%
Often, one has to guard against possible errors by using
the following pattern:
%
\begin{codefile}{sign.ml}
open Unix;;
let tmp cmd sigs = 
\end{codefile}
%
\begin{listingcodefile}{sign.ml}
let old_mask = sigprocmask cmd sigs in 
let treat () = ((* do something *)) in
let reset () = ignore (sigprocmask SIG_SETMASK old_mask) in
Misc.try_finalize treat () reset ()
\end{listingcodefile}

\section{\label{sec/sigsyscalls}Signals and system calls} 

Certain system calls can be interrupted by unignored signals. These
system calls are known as \emph{slow} calls, which can take an
arbitrary amount of time (for example terminal \io,
\indexlibvalue{Unix}{select}, \indexlibvalue{Unix}{system}, \etc).  If
an interruption occurs, the system call is not completed and raises
the exception \ml+EINTR+.  However file \io{} is not interruptible:
although these operations can suspend the running process to execute
another process for disk \io{}, when this occurs the interruption will
always be brief if the disk functions correctly.  In particular, the
throughput of data depends only on the system, and not another user's
process.

Ignored signals are never delivered and a masked signal is not
delivered until unmasked. But in all other cases we must protect our
system calls against unwanted interrupted. A typical example is a
parent waiting for the termination of a child.  In this case, the
parent executes \indexvalue{waitpid} \ml+[] pid+ where \ml+pid+ is the
process id of the child.  This is a blocking system call, it is thus
\emph{slow} and could be interrupted by the arrival of a signal,
especially since the \ml+sigchld+ signal is sent to the parent when a
child process dies.

The module \ml+Misc+ define the function \ml+restart_on_EINTR+ which
makes it possible to repeat a system call when it is interrupted by a
signal, \ie{} when the \ml+EINTR+ exception is raised.
%
\begin{codefile}{misc.mli}
val restart_on_EINTR : ('a -> 'b) -> 'a -> 'b
(** [restart_on_EINTR f x] calls [f x] repeatedly until it does not fail
with [EINTR] *)
\end{codefile}
%
\begin{listingcodefile}{misc.ml}
let rec restart_on_EINTR f x = 
  try f x with Unix_error (EINTR, _, _) -> restart_on_EINTR f x
\end{listingcodefile}
To wait on a child correctly, call
\ml+restart_on_EINTR (waitpid flags) pid+.

\begin{example}\label{ex/childs}
Children can also be recovered asynchronously in the signal handler of
\ml+sigchld+, especially when their return value does not matter to
the parent. But when the process receives the \ml+sigchld+ signal, it
is not possible to know the exact number of terminated processes, since if
the signal is received several times within a short period of time the
handler is invoked only once. This leads to the library function
\ml+Misc.free_children+ function to handle the \ml+sigchld+ signal.
%
\begin{codefile}{misc.mli}
val free_children : int -> unit
(** [free_children signal] free all zombie children of the current process, 
    discarding their exit status. *)
\end{codefile}
%
\begin{listingcodefile}{misc.ml}
let free_children _ = 
  try while fst (waitpid [ WNOHANG ] (-1)) > 0 do () done 
  with Unix_error (ECHILD, _, _) -> ()
\end{listingcodefile}
% 
\ml+free_children+ executes \ml+waitpid+ in
non-blocking mode (option \ml+WNOHANG+) to recover any dead children
and repeats until either there are only live children (zero is
returned instead of a child id) or there are no children (\ml+ECHILD+
exception).

Note that it is not important to guard against the \ml+EINTR+
exception because \ml+waitpid+ is non-blocking when called with the
\ml+WNOHANG+ option.
\end{example}

\begin{example}
The function \ml+system+ in the \ml+Unix+ module is simply defined as: 
%
\begin{lstlisting}
let system cmd = match fork () with
  | 0 -> begin try
          execv "/bin/sh" [| "/bin/sh"; "-c"; cmd |]
         with _ -> exit 127 
         end
  | id -> snd (waitpid [] id);;
\end{lstlisting}
% 
The specification of the \ml+system+ function in the C standard
library states that the parent ignores \ml+sigint+ and
\ml+sigquit+ signals and masks the \ml+sigchld+ signal during the
command's execution.  That makes it possible to stop or kill the child
process without affecting the main program's execution.

We prefer to define the function \ml+system+ as a specialization of the
more general function \ml+exec_as_system+ which does not necessarily 
go through the shell.
%
\begin{codefile}{misc.mli}
val system : string -> process_status
(** [system cmd] behaves as [Unix.system cmd] except that [sigchld] is
blocked and [sigint] and [sigquit] are ignored during the execution 
of the command [cmd] *)

val exec_as_system : ('a -> process_status) -> 'a -> process_status
(** [system exec args] behaves as [system cmd] except that it takes as
arguments a function [exec] and arguments [args] to be passed to [exec]
to start in the new process. In particular, [system] is an abbreviation for 
[exec_as_system (execv "/bin/sh") \[|"-c"; cmd; |\] v]. *)
\end{codefile}
%
\begin{codefile}{misc.ml}
open Sys;;
open Unix;;
\end{codefile}
%
\begin{listingcodefile}[style=numbers]{misc.ml}
let exec_as_system exec args = 
  let old_mask = sigprocmask SIG_BLOCK [ sigchld ] in 
  let old_int = signal sigint Signal_ignore in 
  let old_quit = signal sigquit Signal_ignore in 
  let reset () =
    ignore (signal sigint old_int); 
    ignore (signal sigquit old_quit);
    ignore (sigprocmask SIG_SETMASK old_mask) in
  let system_call () = match fork () with
    | 0 -> 
        reset (); $\label{prog:sreset}$
        (try exec args with _ -> exit 127)
    | k -> 
        snd (restart_on_EINTR (waitpid []) k) in
  try_finalize system_call () reset ();; $\label{prog:stry}$

let system cmd = 
  exec_as_system (execv "/bin/sh") [| "/bin/sh"; "-c"; cmd |];;
\end{listingcodefile}
%
Note that the signal changes must be made before the call
to \ml+fork+ is executed because the parent could receive signals
(\eg{} \ml+sigchld+ if the child were to finish immediately) before it
proceeds. These changes are reset for the child on line~\ref{prog:sreset}
before executing the command. Indeed, all ignored signals
are preserved by \ml+fork+ and \ml+exec+ and their behavior is
preserved by \ml+fork+.  The \ml+exec+ system call uses the
default behavior of signals except if the calling process ignores a
signal in which case it also does. 

Finally, the parent must also reset the changes immediatly after the
call, including if an error occurs. This is why
\ml+try_finalize+ is used on line~\ref{prog:stry}.
\end{example}


\section{The passage of time}

\subsection*{Legacy approach to time}

Since the earliest versions of Unix, time has been counted in seconds.
For compatibility reasons, therefore, one can always measure time in seconds.
The current time is defined as the number of seconds since January 1st, 1970
at \ml+00:00:00+ \textsc{gmt}.  It is returned by the function:

%
\begin{listingcodefile}{tmpunix.mli}
val $\indexlibvalue{Unix}{time}$ : unit -> float
\end{listingcodefile}
%

The \syscall{sleep} system call can pause the execution of a program
for the number of seconds specified in its argument:
%
\begin{listingcodefile}{tmpunix.mli}
val $\libvalue{Unix}{sleep}$ : int -> unit
\end{listingcodefile}
%

However, this function is not primitive. It is programmable with 
more elementary system calls using the function alarm (see 
above) and \ml+sigsuspend+: 

%
\begin{listingcodefile}{tmpunix.mli}
val $\libvalue{Unix}{sigsuspend}$ : int list -> unit
\end{listingcodefile}
%

The \ml+sigsuspend l+ system call temporarily suspends the signals in the 
list \ml+l+ and then halts the execution of the program until the reception 
of a signal which is not ignored or suspended (on return, the 
signal mask is reset to its old value).

\begin{example} Now we may program the \ml+sleep+ function:
%
\begin{codefile}{sleep.ml}
open Sys;;
open Unix;;
\end{codefile}
%
\begin{listingcodefile}[style=numbers]{sleep.ml}
let sleep s = 
  let old_alarm = signal sigalrm (Signal_handle (fun s -> ())) in $\label{prog:sold}$
  let old_mask = sigprocmask SIG_UNBLOCK [ sigalrm ] in
  let _ = alarm s in
  let new_mask = List.filter (fun x -> x <> sigalrm) old_mask in
  sigsuspend new_mask; 
  let _ = alarm 0 in
  ignore (signal sigalrm old_alarm); 
  ignore (sigprocmask SIG_SETMASK old_mask)$\label{prog:ssigproc}$;;
\end{listingcodefile}
%
Initially, the behavior of the \ml+sigalarm+ signal does nothing.  Note
that \quotes{doing nothing} is the same as ignoring the signal.  In
the second instance, the process would not be awakened by the reception
of the signal. The \ml+sigalarm+ signal is put in a non-blocking
state.  Then the process is put on standby by suspending all other
signals which were not already suspended (\ml+old_mask+). After the alarm
is signaled, the preceding modifications are erased. (Note that
line~\ref{prog:ssigproc} could be placed immediately after
line~\ref{prog:sold} because the call to \ml+sigsuspend+ preserves
the signal mask.)

\end{example}

\subsection*{Modern times}

In more modern versions of Unix, time can also be measured in microseconds.
In {\ocaml}, time measured in microseconds is represented by a float.  
The \syscall{gettimeofday} function is the equivalent of the \ml+time+ 
function for modern systems.

%
\begin{listingcodefile}{tmpunix.mli}
val $\libvalue{Unix}{gettimeofday}$ : unit -> float
\end{listingcodefile}

\subsection*{Timers}
In present-day Unix each process is equipped with three timers, each
measuring time from a different perspective. The timers are
identified by a value of type \libtype{Unix}{interval\_timer} :
%
\begin{mltypecases}
\begin{tabular}{@{}ll}
\ml+ITIMER_REAL+ & Real time (\ml+sigalrm+). \\
\ml+ITIMER_VIRTUAL+ & User time (\ml+sigvtalrm+). \\
\ml+ITIMER_PROF+ & User time and system time (\ml+sigprof+).
\end{tabular}
\end{mltypecases}
% 
The state of a timer is described by the \libtype{Unix}{interval\_timer\_status}
type which is a record with two fields (each a \ml+float+)
representing time:
%
\begin{itemize}
\item The field \ml+it_interval+ is the period of the timer.
\item The field \ml+it_value+ is the current value of the timer; 
when it returns \ml+0+ the signal \ml+sigvtalrm+ is sent and 
the timer is reset to the value in \ml+it_interval+.
\end{itemize}
%
A timer is therefore inactive when its two fields are \ml+0+.  
The timers can be queried or modified with the following functions:
%
\begin{codefile}{tmpunix.mli}
type interval_timer = Unix.interval_timer
type interval_timer_status = Unix.interval_timer_status
\end{codefile}
%
\begin{listingcodefile}{tmpunix.mli}
val $\indexlibvalue{Unix}{getitimer}$ : interval_timer -> interval_timer_status
val $\indexlibvalue{Unix}{setitimer}$ : 
    interval_timer -> interval_timer_status -> interval_timer_status
\end{listingcodefile}
%
The value returned by \ml+settitimer+ is the old value of 
the timer at the time of the modification.

\begin{exercise}[noanswer]
To manage several timers, write a module with the following interface:
%
\begin{listingcodefile}{timers.mli}
module type Timer = sig
  open Unix
  type t
  val new_timer : interval_timer -> (unit -> unit) -> t
  val get_timer : t -> interval_timer_status
  val set_timer : t -> interval_timer_status -> interval_timer_status
end
\end{listingcodefile}
%
The function \ml+new_timer k f+ should create a new timer of the 
timer-type \ml+k+ starting the action \ml+f+, inactive at its creation with 
the function \ml+set_timer t+ allowing the regulation of the timer \ml+t+ 
(returning the old value). 

\end{exercise}

\subsection*{Date calculations}

Modern versions of Unix also provide functions to handle dates, see
the structure \libtype{Unix}{tm} which allows dates and times to be
expressed according to a calendar (year, month, \etc) and the
conversion functions: \indexlibvalue{Unix}{gmtime},
\indexlibvalue{Unix}{localtime}, \indexlibvalue{Unix}{mktime}, \etc

\section{Problems with signals}

Owing to their asynchronous nature, the use of signals for inter-process communication 
presents some limitations and difficulties:
%
\begin{itemize}

\item Very little information is transmitted~---~the signal's
type and nothing else.

\item A signal may occur at any point during the execution of the
  program, therefore, signal handling function that access global
  variables must cooperate with the main program to avoid race
  conditions. 

\item The use of signals by the main program entails that long system
  calls made by the program may be interrupted, even if the signals
  keep their default behavior.

\item Reusable functions must always consider the possibility of
  signal use and guard against system call interruption.

\end{itemize}
%

Signals offer only a limited form of asynchronous communication but
carry all the difficulties and problems associated with it. If possible,
it is therefore better not to use them. For example, to wait for a small
amount of time, \indexvalue{select} can be used instead of alarms. But
in certain situations signals must be taken into account (for example in
command line interpreters).

Signals are possibly the least useful concept in the Unix system.  On
certain older versions of Unix (System V, in particular) the behavior
of a signal is automatically reset to \ml+Signal_default+ when it is
received. The signal handling function can of course register itself
again. For example in the \quotes{beep} example on
page~\pageref{ex/beep}, it would be necessary to write:
%
\begin{lstlisting}
let rec beep _ =
  set_signal sigalrm (Signal_handle beep);
  output_char stdout '\007'; flush stdout;
  ignore (alarm 30);;
\end{lstlisting}
%
However the problem is that the signals that are received between the
instant were the behavior is automatically reset to
\ml+Signal_default+ and the moment were \ml+set_signal+ is invoked are
not treated correctly and depending on the type of the signal they may
be ignored or case the process to die instead of invoking the signal
handling function. 

Other flavors of Unix (\textsc{bsd} or Linux) provide better support:
the behavior associated with a signal is not altered when it is
received, and during the treatement of a signal other signals of the
same type are put on hold.


